\chapter{Desenvolvimento}\label{cap:proposta}
Na seção de desenvolvimento deste artigo, exploraremos a proposta deste instrumento de estudo, junto com uma descrição melhor da descrição do método, uma explicação dos resultados esperados e, por fim, o cronograma de execução deste estudo.

\section{Proposta}
O objetivo principal deste trabalho é realizar uma revisão sistemática da literatura sobre o teorema CAP em sistemas distribuídos, a fim de fornecer uma compreensão aprofundada dos princípios subjacentes, dos trade-offs e das estratégias para lidar com os desafios de consistência e disponibilidade, de modo a introduzir novos pesquisadores na área. Os objetivos específicos são:
\begin{itemize}
    \item Apresentar uma visão geral do Teorema CAP, incluindo suas definições e implicações para sistemas distribuídos.
    \item Realizar uma revisão bibliográfica abrangente, buscando artigos científicos, livros, conferências e recursos online relevantes que abordem o teorema CAP e seus aspectos relacionados.
    \item Analisar e comparar os trabalhos encontrados, identificando as principais abordagens e soluções propostas para lidar com os compromissos do teorema CAP.
    \item Investigar estudos de caso e experiências práticas de implementação de sistemas distribuídos, a fim de enriquecer a compreensão dos desafios e soluções associados ao teorema CAP.
    \item Identificar lacunas na literatura e destacar possíveis áreas de pesquisa futura relacionadas ao teorema CAP em sistemas distribuídos.
\end{itemize}

\section{Metologia}
A metodologia a ser adotada para o desenvolvimento deste trabalho será baseada em uma revisão sistemática da literatura. As etapas a serem seguidas são:
\begin{enumerate}
    \item Definição dos critérios de busca: serão estabelecidos critérios de busca claros, incluindo palavras-chave e filtros para selecionar os artigos relevantes para a revisão.
    \item Seleção dos artigos: os artigos serão selecionados com base em critérios de inclusão e exclusão pré-definidos, considerando sua relevância para o tema de estudo.
    \item Análise dos artigos: os artigos selecionados serão analisados em detalhes, identificando os principais conceitos, abordagens, trade-offs e resultados apresentados.
    \item Síntese dos resultados: os resultados serão sintetizados em uma revisão sistemática, organizando os principais pontos discutidos nos artigos e identificando lacunas na literatura.
\end{enumerate}

\section{Resultados e Contribuições Esperadas}
Espera-se que este trabalho proporcione uma visão abrangente dos princípios do teorema CAP em sistemas distribuídos, destacando os trade-offs e desafios associados à consistência e disponibilidade. Além disso, espera-se identificar soluções e abordagens propostas na literatura para lidar com os compromissos do teorema CAP. As principais contribuições deste trabalho são:

\begin{itemize}
    \item Revisão sistemática da literatura sobre o teorema CAP em sistemas distribuídos.
    \item Análise comparativa das abordagens e soluções encontradas na literatura.
    \item Identificação de lacunas na pesquisa atual e sugestão de possíveis áreas de pesquisa futura.
    \item Disponibilização de um artigo de referência para profissionais e pesquisadores interessados no tema.
\end{itemize}

\section{Cronograma de Execução}
O cronograma proposto para a execução deste trabalho segue as etapas definidas na metodologia, todavía é necessário acrescentar as seguintes atividades:

        5. Redação do artigo final.  
        
        6. Revisão e formatação.
        
\begin{center}
\begin{tabular}{||c c c c c c c c c c c c c||} 
 \hline
Atividade & 01/02 & 03/04 & 05/06 & 07/08 & 09/10 & 11/12 & 13/14 & 15/16 & 17/18 & 19/20 & 21/22 & 23/24\\ [0.5ex] 
 \hline\hline
 1 & X & X & X & X & X & X & X & X & X & & & \\ 
 \hline
 2 & & X & X & X & X & X & X & X & X & & & \\
 \hline
 3 & & & & X & X & X & X & X & X & & & \\
 \hline
 4 & & & & X & X & X & X & X & X & & & \\
 \hline
 5 & & & & & & X & X & X & X & X & X & \\
 \hline
 6 & & & & & & X & X & X & X & X & X & X \\
 \hline
\end{tabular}
\end{center}

Este cronograma tem uma duração total de 24 semanas, considerando uma dedicação média de 10 horas por semana.