%%%% GLOSSÁRIO
%%
%% Relação de palavras ou expressões técnicas de uso restrito ou de sentido obscuro, utilizadas no texto, acompanhadas das
%% respectivas definições.

%% Entradas do glossário: \newglossaryentry{rótulo}{informações da entrada}

\newglossaryentry{pai}{%% Informações da entrada
  name        = {pai},
  plural      = {pais},
  description = {um exemplo de entrada pai que possui subentradas (entradas filhas)}
}

\newglossaryentry{componente}{%% Informações da entrada
  name        = {componente},
  plural      = {componentes},
  parent      = {pai},
  description = {um exemplo de uma entrada componente, subentrada da entrada chamada \gls{pai}}
}

\newglossaryentry{filho}{%% Informações da entrada
  name        = {filho},
  plural      = {filhos},
  parent      = {pai},
  description = {um exemplo de uma entrada filha (subentrada) da entrada chamada \gls{pai}. Trata-se de uma entrada irmã da entrada chamada \gls{componente}}
}

\newglossaryentry{equilibrio}{%% Informações da entrada
  name        = {equilíbrio da configuração},
  see         = [veja também]{componente},
  description = {uma consistência entre os \glspl{componente}}
}

\newglossaryentry{tex}{%% Informações da entrada
  name        = {\TeX},
  sort        = {TeX},
  description = {é um sistema de tipografia criado por Donald E. Knuth}
}

\newglossaryentry{latex}{%% Informações da entrada
  name        = {\latex},
  sort        = {LaTeX},
  description = {um conjunto de macros para o processador de textos \gls{tex}, utilizado amplamente para a produção de textos matemáticos e científicos devido à sua alta qualidade tipográfica}
}

\newglossaryentry{bibtex}{%% Informações da entrada
  name        = {Bib\TeX},
  sort        = {BibTeX},
  parent      = {latex},
  description = {um software de gerenciamento de referências para a formatação de listas de referências. A ferramenta Bib\TeX\ é normalmente usada em conjunto com o sistema de preparação de documentos do \gls{latex}}
}

\newglossaryentry{abntex2}{%% Informações da entrada
  name        = {\abnTeX},
  sort        = {abnTeX2},
  see         = {latex},
  description = {uma suíte para \gls{latex} que atende os requisitos das normas da Associação Brasileira de Normas Técnicas (ABNT) para elaboração de documentos técnicos e científicos brasileiros, como artigos científicos, relatórios técnicos, trabalhos acadêmicos como teses, dissertações, projetos de pesquisa e outros documentos do gênero}
}

\newglossaryentry{utfprpbtex}{%% Informações da entrada
  name        = {\utfprpbtex},
  sort        = {UTFPRPBTeX},
  see         = {latex},
  parent      = {abntex2},
  description = {uma suíte para \gls{latex}, baseada na suíte \gls{abntex2}, que atende os requisitos das normas definidas pela Universidade Tecnológica Federal do Paraná (UTFPR), Câmpus Pato Branco, para elaboração de trabalhos acadêmicos}
}
