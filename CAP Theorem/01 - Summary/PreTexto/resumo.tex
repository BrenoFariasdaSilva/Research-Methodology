%%%% RESUMO
%%
%% Apresentação concisa dos pontos relevantes de um texto, fornecendo uma visão rápida e clara do conteúdo e das conclusões do
%% trabalho.

\begin{resumoutfpr}%% Ambiente resumoutfpr
Os sistemas distribuídos desempenham um papel importante na computação moderna, pois permite a criação de softwares escaláveis, flexíveis e de alto desempenho. Contudo, esses sistemas enfrentam instigações inerentes, como a necessidade de prover consistência, disponibilidade e tolerância a falhas de partição em um ambiente distribuído. Com isso em mente, o teorema CAP é uma importante teoria, inicialmente proposta por Eric Brewer, que lida com os trade-offs intrínsecos no design do projeto de um sistema distribuído.
O teorema CAP não é um conceito trivial então, como toda ciência, para ela evoluir, é necessário que o conhecimento desse assunto seja muito bem difundido. Sendo assim, urge a necessidade de compreender profundamente os compromissos atrelados ao teorema CAP, de modo a projetar sistemas distribuídos eficazes e adequados às necessidades específicas de cada aplicação, além de estar ciente das suas limitações.
Este artigo tem como objetivo incluir novos pesquisadores na área de modo a incitar o estudo de novos métodos e soluções. Com isso, o teorema CAP precisa ser apresentado para os iniciantes na área da computação, comparando diferentes resultados e suas implicações em sistemas distribuídos. Dessa forma, é essencial a leitura das principais obras existentes sobre o tema, de modo a promover uma base sólida de conhecimento e estimular futuras pesquisas e inovações nesse campo em constante evolução. 
A pesquisa será conduzida através de uma revisão sistemática da literatura, com base em artigos científicos relevantes sobre sistemas distribuídos e o teorema CAP. Serão considerados estudos de caso e experiências práticas de implementação de sistemas distribuídos para enriquecer a compreensão dos desafios e soluções associadas ao teorema CAP. 
Espera-se que este estudo proporcione um melhor entendimento sobre o teorema CAP e suas implicações na concepção de sistemas distribuídos. Os resultados deverão apresentar uma visão clara dos compromissos envolvidos e das possíveis estratégias para alcançar consistência, disponibilidade e tolerância a partições.
Com base na análise e comparação das abordagens estudadas, espera-se que o artigo conclua que o leitor tenha compreendido por completo o teorema CAP e suas implicações, de modo a estar apto, no processo de desenvolvimento de um sistema distribuído, priorizar critérios de acordo com as necessidades e características específicas de cada aplicação, estando ciente das suas limitações. 
\end{resumoutfpr}
